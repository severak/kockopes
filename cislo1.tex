\usemodule[kockopes]

% definice

\definecolumnset[dvaSloupce][n=2]
\def\qaq {{\ruda \antykwa \noindent Q: }}
\def\qaa {{\ruda \antykwa \noindent A: }}

\definetextbackground[verbatim][background=color,backgroundcolor=lightgray,frame=off,location=paragraph]
\definetyping[C][]
\setuptyping[C][before={\blank[0.5cm]\starttextbackground[verbatim]},after={\stoptextbackground\blank[0.5cm]}]

\kSetupInfo{kolektiv autorů Neslyšného kočkopsa}{Neslyšný kočkopes - číslo 2013/1}

\starttext

\startnotmode[economy]
\startstandardmakeup
\startalignment[center]
\dontleavehmode
\externalfigure[n1/titulka.jpg][height=0.95\textheight]
\title{\tfd Neslyšný kočkopes -- číslo {\os 2013/1}}
\dontleavehmode \externalfigure[n1/minlogo.png][width=5cm]
\stopalignment
\stopstandardmakeup
\stopnotmode

\startmode[economy]
\startstandardmakeup
\externalfigure[n1/minlogo.png][width=5cm]
\title{\tfd Neslyšný kočkopes -- číslo {\os 2013/1}}
\placecontent
\stopstandardmakeup
\stopmode


%%% %%% %%%
{\tfb \iwona \grave \noindent Tvar stanic metra}

Dnes se mě moje sestra zeptala, čeho si všimnu na holce jako prvního
\footnote{Jsou to vlasy a šaty. Takže, dámy, pokud mě chcete zaujmout, mějte zrzavé
vlasy a černobílé šaty. :D}.
Vznikla z~toho docela zajímavá diskuze a~opět se potvrdila skutečnost známá všem
spisovatelům detektivek -- každý svědek vnímá událost jinak.

Kdyby třeba okolo nás prošla hezká holka -- sestra by si všimla, jaké měla boty, sestřin
přítel, jakou měla postavu, a~já, jakou na sobě měla barevnou kombinaci (a~jestli to byla
jedna z těch kombinací, co se mi líbí). Stejně tak jsem si potvrdil, že moje sestra
(stejně jako moje máma) nevidí tvar stanic metra.

Stanice metra v~Praze se totiž dají rozdělit do dvou skupin -- kulaté a hranaté. Je to dáno
způsobem stavby -- ty hranaté jsou hloubené (vykope se jáma, vybetonuje stanice, strop se
zaháže hlínou) a kulaté jsou ražené (vyrazí se tunely a~stanice se staví v~nich). Zjednodušeně
řečeno -- pokud do stanice jedete dlouhým eskalátorovým tunelem, bude mít pravděpodobně stěny
kulaté.

Moje sestra tento rozdíl {\em nevidí}. Nevidí ho do té míry, že mi nedokázala říct, zda je
stanice, kde nastupuje každý den, kulatá, nebo hranatá. A~o~tom to je. Nemá smysl zakládat časopis
o~stanicích metra, to vnímá jen ten, koho to zajímá. Stejně tak nemá smysl zakládat další časopis
o~politice, poezii nebo literatuře. Místo toho je tu Neslyšný kočkopes -- napůl kočka, napůl pes, od
každého trochu. Nevím o~jiném časopise, který by zároveň obsahoval na první pohled tak odlišné věci
jako je báseň a~článek o~programování.

Ale o~tom je život. Holky {\em v úžasnejch botách} (jak by řekla moje sestra) chodí {\em raženými stanicemi
metra} (jak bych řekl spíš já). Jenom každý vnímáme jen tu svou část světa. Schválně, zkuste si vzpomenout,
u kolika lidí, co znáte a potkáváte denně, nevíte, jakou mají barvu očí. No bude jich dost. Na to,
abyste si všimli barvy očí, totiž musíte k člověku přijít docela blízko a podívat se mu do očí.

Doufám, že pro vás bude Kočkopes zajímavý, že se podíváte do očí i~článkům, na něž byste se normálně
nepodívali. Třeba pak zahlédnete i tvar stanic metra…

{\rm \sc Mikoláš Štrajt, zakladatel}

\startnotmode[economy]
\hairline
\placecontent
\stopnotmode

\kStopPrispevku

%%% %%% %%%
\rubrika{Povídky}

\prispevek{Magda Štrajtová}{Konec zvonec světa}

Jmenuji se Pavel Plísemník. Kolik mi je, na tom nezaleží. Teď vám něco povím.
A~ne abyste se mi smáli ani abyste si o~mně mysleli, že jsem blázen.
Já se vám taky nesměju a~nemyslím si o~vás, že jste blázni. Ale jestli
se mi budete smát a~myslet si o~mně, že jsem blázen, mně to může být
úplně ukradený, protože to stejně nebudu vědět. Pro mě za mě, mně to může,
to co si o~mně budete myslet, být ukradený. Stalo se mi to, když jsem jednou
jel vlakem. Cesta trvala hodně dlouho a~já jel sám a~neměl co dělat.
A tak jsem začal přemejšlet o~tom, co mi řekl můj kámoš. Že brzo má
nastat konec světa, že se o~tom všude povídá. Ale nejdříve se musí
splnit pár starých legend.
První legenda zní takto: kdysi dávno ve starém Egyptě jeden starý moudrý muž
předpověděl, že až začne hořet pyramida, nastane konec světa. Jistá žena
v~Indii předpověděla, že konec světa nastane, až se narodí dítě se čtyřma
rukama, jedním okem a~třema nohama. Pak prý to dítě způsobí konec světa.
Ve Francii zase před strašně dlouhou dobou jeden zahradník předpověděl,
že až začnou růst černý kytky, tak nastane konec světa.

Jeden pán, který se jmenoval Mahý Saky, viděl, že konec světa vůbec nenastane,
jenom v~roce asi 2000 se něco bude dít se zemí a~o~13~let později se stane
něco velkolepýho se zemí, a~proto spousta lidí zahyne a~jenom pár lidí přežije.
Jeden Tatar tvrdí, že až tobě, čtenáři, za celý tvůj život se něco ztratí,
nastane konec světa. V~Africe zase zní legenda takto: že až přestane nadýl
svítit slunce, lidé zmrznou a~to bude náš konec světa. Jinde zase tvrdí,
že až někdo vykope poklad Maharadžinců a~najde v~pokladě pozlacenou lebku,
nastane konec světa. Jiná legenda zní, že až na Antarktidě roztajou všechny
ledovce, zaplaví to svět a~to bude potopa a~to bude konec světa.
Na Islandu je legenda, že konec světa si způsobíme sami.
Ale nikde není řečeno jak. Poslední legenda zní, že konec světa nenastane,
aspoň žádnou jinou legendu neznám. A~o~tom jsem přemejšlel celou cestu vlakem.
Když jsem dojel na místo určení, stala se mi divná věc. Měl jsem se setkat se
svým bráchou, ale on tam ještě nebyl, tak jsem na něj musel počkat a~stalo se
mi něco divného. Slyšel jsem, jak si dva lidi povídali. „Představ si, včera
ve zprávach povídali, že v~Egyptě hořela pyramida…“ Ale dál jsem už nic
neslyšel, protože přišel brácha a~jel jsem s ním koupit boty, který brácha
potřeboval, a~pak jsem jel k~němu domů. Když jsem jel s~bráchou koupit boty,
tak jsme jeli autobusem a~já si všimnul plakátu, na kterém bylo napsáno,
že v~Indii se narodilo dítě se čtyřma rukama, jedním okem a~třema nohama,
ale dál jsem ten plakát nestihl dočíst, protože autobus měl už na semaforu
zelenou. Pak když jsme vystoupili a~šli koupit bráchovy boty, šli jsme okolo
květinářství a~já si všimnul, že za výlohou měli vystavený černý růže.
A tak jsem si řekl: „Hořela pyramida, v~Indii se narodilo divné dítě
a~v~obchodě prodávaj černý kytky. Že by se plnilo těch pár věcí, co se
má stát, aby byl konec světa?

Ještě, aby se tobě čtenáři něco ztratilo, ale já si myslím, že se ti, čtenáři,
už někdy něco ztratilo nebo alespoň ztratí. Takže další věc se splnila.“
Pak jsme koupili boty a~jeli domů. Pobyl jsem u~svýho bráchy pár dní
a~pak jsem jel domů. Nic zvláštního se nestalo. A~když jsem dojel domů,
tak mě navštívil můj kámoš a~povídá, že byl v~Africe na návštěvě u~svého
strýčka a~ten mu řekl: „Představ si, že v~Africe je město, co se jmenuje
Slunečný. Teda se mu říká spíš tmavé, proč, to ti hned povím. Stalo se to
asi takto -- před strašně dlouhou dobou tam přestalo svítit slunce a~strašně
dlouhou dobu tam nesvítilo, asi tak půl roku. Spousta lidí měla málo síly
a~byli nemocní. Ale nikdo neví, proč nesvítilo, a~taky tam bylo místo tepla
najednou zima.“ Tak jsme si dlouho ještě povídali, a když odešel, bylo už
pozdě večer. A tak jsem šel spát, ale ještě předtím jsem se kouknul na zprávy.
Ve zprávách byly jako vždy samé katastrofy, ale jedna věc mě zaujala.
Zaujalo mě toto: že v~jedný vesnici, která se jmenuje Paví Oko,
se od roku 2000 dějí samé katastrofy, třeba čtyřikrát do jednoho roka tam
byly povodně nebo asi 100 bouraček za 13 let nebo si tam prdlo hodně lidí
najednou a~pět lidí se udusilo nebo výbuch letadla a~tak dále. Ale skoro celá
vesnice vymřela nebo se odstěhovala, ale o~13 let později se tam začalo dít
čím dál míň katastrof. A~pak jsem to vypnul a~šel spát. Ráno jsem se probudil,
a~protože byla sobota, tak jsem si udělal snídani a~zapnul televizi.
Zrovna tam běžel dokument o~Antarktidě. A~zrovna tam říkali, že ledovce
tajou hodně rychle, a pokuď to takto bude pokračovat, tak brzo roztajou.
Ale najednou se tam objevily barevný pruhy a~zrnění, a~když to konečně
naskočilo, tak se tam objevil nápis: »Tato porucha není na vašem příjmači«.
Tak jsem vypnul televizi a~šel si číst noviny a~titulek na nich byl,
že Duga Pijavica vykopal poklad Maharadžinců. Vy asi nevíte, kdo to je,
tak já vám to povím. Je to slavný archeolog. A~on v~tom pokladě našel
pozlacenou lebku. Pak jsem si uvědomil, že se už splnily všechny ty legendy,
co mi můj kámoš řekl, že se maj splnit, aby se stal konec světa.

Tak jsem si sednul do kouta a čekal na konec světa.
Ale žadný nepřicházel, a tak jsem si řekl, že to, co se má stát,
aby byl konec světa, se nestalo. Už jsem si oddechnul, že se mi nic nestalo,
ale najednou se mi před očima objevil nápis:

{\rm \sc Konec zvonec světa}

\kStopPrispevku

\kStrankaIlustrace{n1/konec-zvonec.jpg}{„Hořela pyramida, v~Indii se narodilo divné dítě
a~v~obchodě prodávaj černý kytky. Že by se plnilo těch pár věcí, co se
má stát, aby byl konec světa?}

%%% %%% %%%
\prispevek{Anna Benešová}{Otevřená lekce}

„Když ty jsi taková… Všechno řešíš otevřeně a to já neumím…“

To byla tvá poslední slova. S těmi na rtech jsi odešel. Měl jsi pravdu. Výjimečně. Od chvíle, co jsme se poznali, jsem tě chtěla. Líbilo se mi na tobě všechno. Tvoje nešikovnost, neschopnost flirtovat, nervozita v mé přítomnosti hraničící s~posvátnou hrůzou, neboť jsem žena. Dobrý postřeh, miláčku, vskutku jsem žena. Mám vše, co k ženě patří, a ještě nějaký zdroj vybuchující energie z neznámého zdroje, který mi občas velí vzít chlapa za vlasy a odtáhnout do postele.

Změnila jsem tě? Asi ano. Býval jsi vyděšený kluk. Vedle mě se z tebe stal chlap. Učila jsem tě, že svět je ošklivý a musíš bojovat, aby ses v něm neztratil. Udělala jsem z tebe silného ambiciózního muže. Tak silného, že jsi měl sílu mě opustit.

Myslím, že byl problém v tom, že jsem na tebe byla moc hodná. Ty sis toho nevážil. Řekni, kde bys našel takovou, jako jsem já? Tak třeba ta tvá nešikovnost v sexu. Na začátku byl tvůj výkon (lze-li to vůbec nazvat výkonem) jednoduše tristní. Netušil jsi, co máš dělat s tím, co ti Bůh nadělil do kalhot. On se taky nepředal, že? Já měla svatou trpělivost. Postupně jsi nabyl trochu sebevědomí. Nikdy jsem nepředvedla tolik hereckých etud jako při předstírání orgasmů.

Přesto všechno jsem tě neopustila. Jednou jsem si tě vybrala a milovala tě. To nejde vzít zpět. Milovala jsem tě celým svým srdcem a celou svou duší. Tak to má být. Proto jsem tě nemohla nechat odejít. Láska se ze všeho nejvíc blíží nenávisti. Ta má výbušná energie se proměnila v destruktivní sílu a velela ničit zdroj mé bolesti. Rozmetat na kousky toho, kdo mi tak ublížil. Neměla jsem na výběr. Stál jsi tam s tím svým kufříkem na kolečkách a mluvil jsi o tom, že už mě nemiluješ, že to asi byla chyba…

Co byla chyba? Náš život? Naše láska? Kdy tě to napadlo? Máš jinou? V hlavě jsem měla tisíc otázek, které se slovy, které ti proudily z krásných úst, pomalu splynuly v jednu černou tmu. Nic jsem neviděla, jen cítila. Cítila jsem vztek. Obrovský vztek. Moje ruka nejednala na můj pokyn. Jednala za mě. Psychické a fyzické já se spojilo v jediném pohybu, který jsem v krátkém časovém intervalu opakovala. Nejdříve jsem si neuvědomila následky. Docela mě to bavilo. Pak jsem se vyděsila. Řekla jsem ti, že mě mrzí, že jsem ušpinila dlaždice, a že to uklidím. Ty už jsi tam nebyl. Byla tam jen politováníhodná hromádka něčeho, co dříve byl živý člověk. Nechápala jsem, že ty končetiny, hlava a krev, všechny ty součástky bez hlubšího smyslu, mohly někdy tvořit tebe. Přišlo mi docela legrační, že jsi mluvil o mé otevřenosti vůči světu, která tě ničí, a já teď otevřela tvoje tělo. Doslova. Myslela jsem si, že odněkud přijdeš a společně se tomu oba zasmějeme. Nepřipadá ti legrační, že tě tenhle kus masa připomíná?

Už asi nepřijdeš. Uvařila jsem si kávu a uvažuji, co udělám s tvým tělem. Nemám nejmenší chuť strávit zbytek života v kriminálu. Musím se zbavit těla. Asi je to ode mě trochu sobecký postoj, ale neměl jsi mě chtít opustit. Nakonec jsi ale dosáhl svého, viď? Jsme od sebe odloučení… Do konce života určitě. Tedy teď to znamená do konce mého života… 

\kStopPrispevku

\kStrankaIlustrace{n1/otevrena-lekce.jpg}{Nechápala jsem, že ty končetiny, hlava a krev, všechny ty
součástky bez hlubšího smyslu, mohly někdy tvořit tebe.}

\doifmode{economy}{\page}

%%% %%% %%%
\prispevek{Mikoláš Štrajt}{Zázvor aneb Malá volební hudba}

\kCitat{Jednu věc si ale uvědom: Ďasa si nevymejšlí pánbu. Čert se dycky rodí z lidského hříchu, i když se ho lidi snažej zapudit. Jeden knihou, jinej ohněm, další eště něčím jiným... No a když ho konečně zažehnaj, a že se s tím nějak nějak natrápěj, tak skrze to svý trápení zploděj dalšího čerta -- a zase to všecko běží vod začátku pěkně dokola....}{S. Jaroslavcev: Ďábel mezi lidmi}

Učitel je pro studentky alfa samec.

Až se budete někdy rozčilovat nad podobným článkem, buďte v pozoru. Je to tím, že ho napsal učitel. Stejně tak až vás budou plakáty po celém městě přesvědčovat, že všichni volí Taškára, vězte, že byl Taškár zamlada, předtím než se dal na politiku, malíř, a všechny ty kreativce tak proto učili jeho spolužáci. A pak vás nemusí překvapit, že Kozák vyhrává průzkumy veřejného mínění. Studoval sociologii a absolventi sociologie, právě ti ty průzkumy dělají.

Umělci dostávají místa v reklamě a sociologové falšují průzkumy trhu. Kam ten svět spěje?

Stejně tak není důvod, proč důvěřovat městské lince vlaku. Vede ve směru, kam by normální člověk vlak nikdy neposlal. Proto většinou slouží train abusingu (nehodlám tu teď vysvětlovat, co to je, sám to totiž často provozuju) a převozu podivných existencí z Jižního nádraží na Východní a zpět.

Markéta seděla a učila se na zkoušku z biologie o nějakých absurdních breberkách žijících ve slaném nálevu v továrně na kyselé okurky. Na Jižním nádraží přistoupili dva starší pánové, asi měli trochu upito. Ale to nebylo nezvyklé, tímhle posledním vlakem nikdo střízlivý nejezdí.

„Večer si doma dám,“ začal jeden. „Jak se tomu říká? Babička to dávala mě malýmu -- takový kořínek je to....“ -- Druhý vzpomínal: „Jo nedávno jsem si z toho dělal čaj...“ -- „Jak vono to jen....“

„Zázvor“ řekla do ticha Markéta. Kyselých okurek měla už po krk.

„To je ale chytré děvče,“ pochválil ji jeden, „zázvor je to.“ -- „Ano. Ona je budoucnost tohoto města, my už jsme udělali dost,“ přizvukoval druhý.

Dvořit se uměli asi jako bojové vozidlo pěchoty náramkovým hodinkám a nejhorší na tom bylo to, že už ji dneska s tím, že jí to sluší, zastavoval pro změnu bezdomovec (jménem Leopold, který teď seděl na Jižním nádraží a kouřil cigarety značky Drina, které vyžebral od jednoho Jugoslávce).

„Když mi bylo dvacet, jezdila touhle tratí jedna moc pěkná průvodčí,“ mlel si svoji písničku první.

Markéta vzhlédla od nálevníků ve svých skriptech. Odkud jen ty dva dědky znala?

Vlak projel okolo domu, kde byl plakát jednoho z nich v nadlidské velikosti.

No jistě! To jsou kandidáti na starostu! Už 14 dní tu bojujou o hlasy lidí jak dva rozzuření nosorožčí samci a teď je vidím vedle sebe družně hovořící ve vlaku. No potěš koště! 

Vlak vjel na most.

„Věřím jen těm statistikám, které jsem si sám zfalšoval, jak říkaval Churchill…“ poučoval druhý. „A nebo to říkal Goebbels? No, to je jedno, na tom nesejde. Každopádně si myslím, že máš tohle období výhru skoro v kapse. Z demografického hlediska jsi zacílil na generaci babyboomu…“

Markéta vzhlédla od nálevníků. Kozák a Taškár, ale který je který? Navíc mají prohozené insignie: Ten s odznáčkem má na klopě obličej toho druhého, toho co něco zkoumá v brožuře „Společně za město“ (neveděla, že to, co ho zaujalo, byla jedna hezká fanynka té strany, ve skutečnosti začínající irská herečka). Oba politikové se tvářili nadmíru spokojeně. Ta noc byla jejich. Sčítání hlasů začalo. Vlak vjel do tunelu.

Ona taková okresní politika, když na to přijde, je leckdy drsnější než politika mezinárodní.

Bezdomovec Leopold dokouřil poslední Drinu. Nechápal, proč se kolem těch posledních voleb strhl takový povyk. Co on pamatoval, byli na radnici oba už několikrát. Nic se za tu dobu nezměnilo, akorát se postavil silniční obchvat a jedna tramvajová trať. Ani jeden z kandidátů se ve svém programu o bezdomovcích nezmínil. Dokonce je ani nechtěl vyhnat z centra, jak bývalo při minulých volbách zvykem.

Jak tak filozofoval, objevila se stará pomalá postava. Sekuriťák. „Musíte odejít,“ řekl klidně. -- „Ale kam?“ položil Leopold řečnickou. -- „To nevím, třeba na radnici, ale musíte odejít.“

Leopold vstal a v doprovodu sekuriťáka vyšel z nádraží. Sotva vyšli, dveře nádraží se zavřely a celá ta budova zajela za tichého hučení kamsi pod zem. „Generální úklid,“ vysvětlil sekuriťák.

Markéta šplhala po schodech domů. Celý život bydlela na kopci nad městem, kde příšerně foukal vítr a bylo tam vždy o deset stupňů míň než v jakékoliv jiné části města, počítaje v to i akciové mrazírny. A uprostřed schodů, kudy chodila domů, stála na kamenném soklu obrovská prázdná bronzová mísa. Už od dětství měla Markéta jasno v jejím účelu: Určitě slouží k obětování lidí.

Nicméně až do onoho předvolebního večera nikdy nebyla svědkem nějakého krvavého obřadu. Až dnes -- nějaká okabátovaná postava tu v míse cosi pálila a zapíjela to nějakým držkopalem. Markéta dostala strach. To poslední, co potřebovala, bylo střetnout se s nějakým šíleným satanistou či co. Věděla totiž, že právě ONA je typ vhodný pro zápalnou obět.

S ohněm v očích stoupala dál, ale ta postava jí byla čím dál znamější. Byl to jejich profesor praktické filozofie!

On to byl teda divný patron. To jo. Divný dokonce i na poměry filozofické fakulty. Ale tohle?

„Dobrý den!“ pozdravila ho schválně dost nahlas. Vzhlédl od bronzové mísy: „Dobrý večer, Markéto, co tady děláte?“ -- „Na to samý jsem se chtěla zeptat já vás. Já tu totiž bydlím.“ -- Nadychl se, položil poloprázdnou láhev Zázvorovice na zem vedle mísy a podíval se na Markétu: „Dělám takový malý obřad na ozdravení demokracie.“ -- Skoro vyprskla smíchy: „Prosím!?“ -- „No prostě pálim nepoužité volební lístky za tichého zpěvu Finských metalových balad…“ Na tváři Markéty se začal rýsovat jakýsi neurčitý škleb, tak spěšně dodal: „Je to taková moje tradice, dělám to každé volby.“ -- „Tak doufám, že se vám tu demokracii podaří ozdravit.“ Řekla s trochou ironie v hlase. „Viděla jsem před chvílí Kozáka s Taškárem, jak se spolu družně baví.“ -- „To jste nevěděla?“ zeptal se jí překvapeně.

Ne, nevěděla to. O politiku se nezajímala a to málo, co se dozvěděla, jí vždy stačilo utvrdit v jejím nezájmu. Netušila, že se znají už z dob před svým politickým působením. Netušila, že mezi jejich stranami se vede už témeř sto let taková místní studená válka. Nesnáší se, ale potřebují se. Jedna strana si nemůže dovolit druhou zlikvidovat, protože by se pak neměla vůči čemu vymezit.

Rozloučila se s filozofem a pokračovala domů. Bydlet sám má své nepochybné nevýhody a pochybné výhody. A naopak. To, že mohla ve dvě v noci v klidu šramotit (ale i rachotit a rámusit) po bytě, byla ta pochybná výhoda. To, že všechny talíře, co ráno nechala nemyté v kuchyni, stále leží tam, kde je položila, byla ta nepochybná nevýhoda. Byla však příliš ospalá na to, aby s nimi něco udělala.

Už stála v koupelně v županu a čistila si zuby, když si na něco vzpomněla. Volební lístky!

V zásuvce psacího stolu ležela ještě neotevřená sada volebních lístků. Markéta nezvolila\footnote{A tady je dobré připomenout, že v Hlubočici mají místní (obecní) úřady o dost silnější pravomoci než v Čechách. Jedním z důsledků je i přímá volba starosty.}.

Jedním z~důvodů byl mizerný výběr. Z~kandidátů připadala v~úvahu asi jen matka její spolužačky Valérie, samotná Valérie se však na možnost, že by se přes noc stala dcerou starostky, dívala docela kriticky. Kromě toho Markéta neměla čas. Všechny ty výstavy, zkoušky, včerejší koncert Obwodu LRC…

Otevřela zásuvku a~vzala lístky. Přemýšlela, zda to stojí za to chodit ven, a~nakonec se rozhodla, že jo. Oblékla si mikinu a~tepláky, na šaty, ve kterých přišla, už byla venku moc zima.

Oheň dohoříval. Filozof seděl na okraji mísy a~zamyšleně pozoroval v~ohni zániku se kroutící lístky. -- „Pane profesore,“ oslovila ho, „něco jsem vám přinesla.“ A nabídla mu obálku s lístky. -- „Děkuju vám.“ Přijal lístky a na znamení vděčnosti udělal lehce komickou poklonu. Roztrhl obálku a rozházel její obsah do dohořívajícího ohně.

Dobrovská, Kozák, Taškár, Flink… Jména se kroutila a měnila v prach. Markéta koukala do ohně a přemýšlela, zda to není škoda. Zda to není škoda papíru nebo jejího hlasu. Přeci jen se stala součástí zápalné oběti!

Přisedla si na mísu a zadívala se do ohně. Filozof se pomalu zvedl. Zřejmě mu nepřipadalo vhodné sedět na jedné míse společně se svou studentkou. Nehleděl na to, že ta mísa má tři metry v průměru.

„Markéto, víte proč se pořádají volby?“ -- „Ne,“ odpověděla prostě. Na noční diskuzi spojenou s citováním Nietzscheho a Platóna neměla ani čas ani chuť ani náladu. -- „Z toho samého důvodu, jako se ve starém Římě pořádaly kalendy. Lidé musí na pár dní získat pocit, že je vše dovoleno. Že si o svém životě rozhodují sami, a považte to -- dokonce, že si vládnou sami!“ Filozof domluvil. Dole v údolí projel po železničním mostě nákladní vlak. Markéta mlčela.

„Hmmm. A po volbách…“ Pokračoval filozof a vzal jeden z volebních lístků z mísy. Lístek se mu tím zvednutím v ruce dočista rozsypal na popel. „A po volbách se tahle iluze rozsype. Skutečné moci se opět chopí nikým nevolené šedé eminence a další čtyři roky novodobého otroctví se rozjedou nanovo.“ Vlak v údolí vyjel z tunelu a plazil se jako železný had mezi odstavenými vagóny na Východním nádraží.

Markéta vzala láhev Zázvorovice stojící u paty mísy a ochutnala. Otřásla se jako pes. Zázvorovice bylo vůbec to nejšílenější, co kdy pila. Nad parkem, kde stáli, projel taxík.

„Jak můžete s takovýmhle názorem každé ráno vstát, jít do školy a tam vyučovat?“ zeptala se ho a položila lahev Zázvorovice na okraj mísy. Filozof lovil v hlavě jakýsi Nietzscheho citát, ale než ho stačil vyslovit, Markéta zmizela za teréním zlomem v přízemní chodbě činžáku.

{\sc \rm Ámen}

\kStopPrispevku

\kStrankaIlustrace{n1/zazvor.jpg}{Dvořit se uměli asi jako bojové vozidlo pěchoty náramkovým hodinkám.}

%%% %%% %%%
\rubrika{Recenze}

\prispevek{David Hromádka}{Ester Krejčí}

Již delší dobu mnozí kulturně vzdělaní lidé vyjadřují znechucení nad
dekadencí soudobých populárních filmů a~seriálů, které diváky podporují
v~konzumním způsobu života a~zásobují je snadno vnímatelnými, leč bezobsažnými,
slastmi, jejichž přemíra má pak negativní vliv na osobnost
diváků (zvlášť mladých) a~potlačuje jejich přirozenou kreativitu
a~schopnost sebereflexe.

Naštěstí ještě vznikají i~díla, jejichž prvotním účelem není působit divákům
slast při své pasivní konzumaci, ale něco vypovědět, rozšířit divákovi obzory,
pomoci mu poznat náš svět z~nových úhlů pohledu a~snad i~podnítit úvahy
nad sebou samým či nad soudobou společností a~kulturou. Seriál Ester Krejčí
k~pokusům o~takové dílo patří.

Seriál začíná v~pokoji Jany Zlomilové, celkem běžné, leč nesmělé a~ostýchavé
osmnáctileté dívky. Na sobě má úbor, jež (až na chybějící kapuci) nápadně
připomíná známý myší kožíšek z~pohádky Princezna se zlatou hvězdou.
Vydává se poprvé do nové školy, a~zatímco řeší svůj potenciálně banální
problém -- jak zapadnout do tamějšího kolektivu, skrytý vypravěč divákovi
vysvětluje, proč si dcera vyká i~s~vlastními rodiči a~proč je Nikola Merlinová
jejich třídní profesor, ne profesorka. Jazykové odlišnosti nečiní dílo nijak
nesrozumitelným, divák si však při srovnání jazyka postav s~naší češtinou
může uvědomit míru, do jaké je náš jazyk svázán potřebou vyjadřovat pohlaví
člověka, ale současně také, že jde o~umělý a~snadno překonatelný kulturní
konstrukt.

Vraťme se však k~Janě Zlomilové. Ukazuje se, že její problém nebude tak
triviální, jak by se dalo očekávat. Z~nových spolužáků nemá dobrý pocit,
už od počátku jí připadají divní (v~tom se s~ní zřejmě většina diváků shodne).
Zvlášť jí zaujme Ester Krejčí, která ji docela protivně odbude,
a~Moje vinná réva, která je jí podezřelá už svojí přezdívkou.
Ve třídě se k~ní první den nikdo spontánně nehlásí.

Zatímco ve škole je Jana Zlomilová zamlklá a~nesmělá, doma se rozvalí na
postel, vytáhne mobilní telefon a~vypoví svoje dojmy příteli Vlastě
Novotné (to nepřechylování je míněno vážně a~diváci si na něj dříve či
později zvyknou).

Druhý den ve škole Jana Zlomilová překoná svoji nesmělost a~odhodlá se
se spolužáky seznámit. Její původní dojem se víceméně potvrzuje.
Spolužáci jsou divní. Jeden z~nich se k~ní dokonce zachová tak
nechutně, že to Jana Zlomilová nevydrží a~uteče na záchod.
Ukáže se však, že Moje vinná réva není taková, jakou si ji Jana Zlomilová
představovala; chce jí pomoci a~vybízí k~tomu i~Ester Krejčí.
Ta však její entuziasmus nesdílí a~jejich vzájemný konflikt
nabírá pomalu na absurditě, což si Moje vinná réva uvědomí,
a~tak sama nabídne Janě Zlomilové své přátelství, čímž nastalý problém
vyřeší. Děj však nekončí, jen je přerušen hodinou vyučování, což je další
ujištění, že navzdory nezvyklým způsobům chování a~jazykovým odlišnostem
se příběh neodehrává
v~žádné předaleké galaxii, ale na celkem obyčejném pozemském gymnáziu,
jaké většina diváků snad měla příležitost zažít. Po hodině Moje vinná réva
citlivě seznámí Janu Zlomilovou s~dalšími spolužáky a~objeví se první náznaky
soucitu. Noví spolužáci už pro Janu Zlomilovou nejsou takoví divní, obávaní
a~nepochopitelní nepřátelé, jaké si původně představovala, ale jsou mezi nimi
i~takoví, se kterými si trochu rozumí a~domluví se s~nimi.

Když se pak Jana Zlomilová vrátí domů, už ví, že nová škola nebude tak špatná,
jak si myslela, a~myšlenka na to, že se spolužáky stráví dalších téměř
devět měsíců do maturity, už není nepříjemná. Epizoda končí, ale divákovi
je jasné, že to je teprve začátek, a~píseň Within Temptation o~tom,
že všichni jsme součástí nekonečného příběhu, která hraje na pozadí titulků,
to jen potvrzuje.

Děj první epizody je celkem jednoduchý a~přímočarý (až na eukatastrofu
způsobenou Mojí vinnou révou). Vypráví o~příchodu vcelku obyčejné dívky
na nové gymnázium -- do třídy plné hochů a~dívek budících negativní první
dojem. Nekomplikovanost děje umožňuje divákovi soustředit se na charaktery
postav, které jsou vzhledem k~jejich velkému množství a~malému rozsahu epizody
většinou jen naznačeny (rozvinuty budou v~dalších epizodách).
Zatímco Jana Zlomilová a~Moje vinná réva mají šance získat si sympatie diváků
velmi rychle, ostatní dvě hlavní postavy, Ester Krejčí s~až nelidsky chladným
vystupováním a~Tamara Janů, která působí jako zdrogovaná
(zvlášť když podezřívá Janu Zlomilovou, že pláče na záchodě proto, že tam
krájí cibuli), si na to počkají
alespoň do příští epizody. Také si lze povšimnout, že divák si začne zvykat
na vykání mezi spolužáky, které v~dnešním českém kulturním kontextu působí
značně nezvykle, a~již zmíněné nepřechylování, které by si dnešní feministky
nejspíše vyložily jako příklad útlaku žen, ale ve skutečnosti jde spíše
o~podnět k~delší a~hlubší analýze či polemice.

Seriál Ester Krejčí bohužel dosud nebyl zfilmován, scénáře všech
vydaných epizod jsou však k~dispozici na stránkách \goto{http://esterkrejci.xf.cz/}[url(http://esterkrejci.xf.cz/)]
v~sekci {\em Epizody}. Snad je to i~dobře, neboť žádný režisér nedokáže
natočit tak krásný seriál, jaký si vy jako čtenáři dokážete představit.
Myslete na to, až budete číst scénář, a~jistě si odnesete řadu nových
zážitků.

\kStopPrispevku

\kStrankaIlustrace{n1/ester-krejci.jpg}{Tamara Janů, která působí jako zdrogovaná
(zvlášť když podezřívá Janu Zlomilovou, že pláče na záchodě proto, že tam
krájí cibuli)}

%%% %%% %%%
\rubrika{Programování}

\prispevek{David Hromádka}{Tipy, triky a~vtipy v~C++}%    Nadpis
%
V~této pravidelné rubrice se budete dozvídat o~tajuplných zákoutích
podivuhodného multiparadigmatického programovacího jazyka C++. Rubrika
předpokládá dobrou znalost tohoto jazyka, takže pokud nemáte tušení,
za jakých okolností se pole automaticky konvertuje na ukazatel na svůj
první prvek, tato rubrika nebude pro vás. Pokud se to chcete dozvědět,
mohu doporučit knihu {\em Jazyky C a~C++ : kompletní průvodce},
jejímž autorem je Miroslav Virius.

Dnes poodhalíme tajemství polí a~jejich rozměrů. V~řadě učebnic najdete
radu uschovat si rozměr pole do celočíselné konstanty:%

\startC
const int pole_rozmer = 4;
int pole[pole_rozmer] = {1, 2, 3, 4};
\stopC

Toto řešení má ale dvě nevýhody. Především si musíte zapamatovat název pomocné
konstanty. Větší problém ovšem nastane, když se rozhodnete nechat si rozměr pole
odvodit podle počtu inicializátorů:

\startC
int pole[] = {1, 2, 3, 4};
\stopC

V~tomto případě je totiž uvedené řešení nepoužitelné. První další řešení,
které většinu programátorů napadne, je následující použití operátoru
{\tt sizeof}:
\startC
sizeof(pole) / sizeof(int)
\stopC
Není to úplně špatná myšlenka, ale má další dva nedostatky. Předně když se
rozhodnete změnit typ prvku pole například na {\tt double},
začne uvedený výraz dávat chybné výsledky. Tento nedostatek lze odstranit
drobnou úpravou:

\startC
sizeof(pole) / sizeof*(pole)
\stopC

Oba uvedené výrazy jsou ovšem náchylné k~chybnému použití. Například:

\startC
int funkce(char pole[20]) // pozor, není pole!
{
    return sizeof(pole) / sizeof*(pole);
}
\stopC

Uvedená funkce téměř jistě nevrátí číslo 20, ale spíš 4, nebo 8. Její parametr
je totiž navzdory syntaxi ukazatel, ne pole.

Jednoduché, bezpečné a~rozšiřitelné řešení nabízejí šablony:

\startC
template<typename T, size_t SZ>
struct pole_type {
  typedef T type[SZ];
};

template<typename T, size_t SZ>
typename pole_type<char, SZ>::type &arraydim(
  T (&)[SZ]) throw ();
\stopC

Funkce arraydim() transformuje libovolné pole na referenci na pole znaků
o~stejné dimenzi. A~protože podle standardu {\tt sizeof(char) == 1},
můžeme nyní dimenzi pole zjistit výrazem:

\startC
sizeof arraydim(pole)
\stopC

Mechanismus šablon zajišťuje statickou typovou kontrolu a~přetížením funkce
arraydim() můžeme její funkcionalitu rozšířit o~podporu dalších typů,
například bitových sad:

\startC
template<size_t SZ>
typename pole_type<char, SZ>::type &arraydim(
  const std::bitset<SZ> &) throw ();
\stopC

Protože funkci arraydim() budeme používat jen uvnitř operátoru
{\tt sizeof},
nemusíme definovat její tělo, což ušetří práci kompilátoru, protože ji
nebude muset kompilovat. Konstrukci {\tt sizeof arraydim} pak
můžeme zabalit do makra preprocesoru, ale to už nechám na vás.

\emptylines[1*halfline]
\noindent A~malý vtip na závěr:

\startC
for (size_t i = 8; i >= 0; --i)
  printf("Shall I end? Nevermore!\n");
\stopC

\kStopPrispevku

%%% %%% %%%
\rubrika{Úvahy}
\prispevek{David Hromádka}{Myšlenky s ručením omezeným}

Následující myšlenky nemusejí být v žádném smyslu pravdivé. Předkládám
vám je jako výplody své duše bez jakékoliv záruky, dokonce bez implicitní
záruky pochopitelnosti. Jsou to myšlenky s ručením omezeným.

\startitemize
\item Pochází slovo myšlenka ze slova myš, nebo naopak?
\item Mimozemšťané jsou produktem lidské fantazie podobně jako například
draci. To ovšem neznamená, že neexistují.
\item Kruhy v obilí jsou většinou malé a nemají vůbec kruhový tvar. Pozornost lidí však přitahují jen ty velké a kulaté.
\item Většina toho, co dnes lidé říkají, má sexuální podtext. Pro většinu lidí
je nepříjemné myslet na věci, které takový podtext nemají.
\item Je rozdíl mezi významy slov ezoterický a exoterický ezoterický, nebo
exoterický?
\item Bolí lidi hlava z myšlenky, zda je horší brát drogy, nebo se za to stydět?
\item V této verzi knihovny není zřejmé, co by měla tato funkce dělat. Proto
je doporučeno ji zatím příliš nepoužívat.
\item Jestliže to není cokoliv jiného než kachna, je to kachna. Jestliže to není
kachna, je to něco jiného.
\item Jestliže černá díra je analogií smrti, dochází i u smrti k dilataci času?
\item Kdyby existence neexistovala, nebylo by smyslem existenciální filozofie
hledat smysl své existence? Nesmysl? Neexistence?
\item Přirozené jazyky jsou regulární. Jejich vyjadřovací síla je ekvivalentní
výpočetní síle nějakých konečných stavových automatů.
\item Lidé, kteří nejsou programátoři, mají problém (s chápáním závorek
(jen těch vnořených (od určité úrovně vnoření))).
\stopitemize

\kStopPrispevku

\kStrankaIlustrace{n1/myslenky-sro-2.jpg}{Většina toho, co dnes lidé říkají, má sexuální podtext.}
\kStrankaIlustrace{n1/myslenky-sro-1.jpg}{Pro většinu lidí je nepříjemné myslet na věci, které takový podtext nemají.}

%%% %%% %%%

V Čechách existuje několik diskuzních evergreenů, kde je každá další diskuze jen přiléváním leteckého benzínu do už tak dobře hořícího ohně. A právě jednomu z těchto „hořlavých“ témat bych rád věnoval tuto úvahu.

\prispevek{Mikoláš Štrajt}{Romské otázky}

\startblockquote
Pokud čelíte lidem na ulici, kteří nadávají na cikány, je dobré se jich zeptat:
A vy byste se chtěl narodit jako Rom, s tmavou pletí, do sociálně vyloučené
lokality?
\stopblockquote

Tuto poněkud provokativní otázku jsem po přečtení jednoho článku nahodil
na svou facebookovou zeď a čekal. Čekal jsem, zda se někdo chytne.

A chytla se jedna slečna. Na mou otázku sice neodpověděla, nicméně
mi položila do komentáře seznam otázek, které celkem odpovídají rozšířeným
myšlenkám o~Romech:

\startblockquote
A proč jsou tak hluční?
Proč nechtěj makat, jako všichni ostatní?
Proč jen sajou prachy ze státu?
Proč se jich každej zastává a na bílí se sere?
Když napadne bílej Roma, tak z~toho dělaj nevim co,
ale když je to naopak, tak se neděje nic?
Tohle je absolutně nesprávný. Jo, a jsem rasista!
Nikdo si na mě nikdy nedovolil to, co cikáni!
Vidím partu mladejch cikánů a vím, že bude problém…
\stopblockquote

Rozhodl jsem se tedy pojmout tyto otázky jako osnovu své slíbené úvahy.
Nuže -- otevírám kanystr a~jdu přilévat benzin do ohně svými odpověďmi.

\qaq A proč jsou tak hluční?

\qaa To je dost široce položená otázka. Nejdříve je nutné upřesnit, jak hluční.
Každá hlučnost má totiž jiný důvod. Můžou na sebe například hulákat na ulici,
můžou do noci „oslavovat“, jejich výrostci si můžou velmi hlasitě pouštět
hip-hop. Za hlučnou můžeme konec konců považovat i~cikánskou cimbálovou
kapelu, když nás shodou okolností ruší ze spánku.

Odpovím protiotázkou: A proč jsou bílí tak hluční?

Vždyť co jiného než hluk je to, když se v první slunečný jarní den
po zimě rozjedou na všech zahradách okolo sekačky, na některých dokonce dvě?
Co jiného než hluk je hučící venkovská diskotéka ve tři ráno v noci?
Co jiného než hluk je noční vyřvávání bílých opilců?

A~tady jsem to, myslím, ukázal jasně:
Hluk souvisí s etnicitou člověka jen naprosto
minimálně\footnote{i~když se říká, že například Italové jsou hlučnější
než třeba Poláci nebo Japonci. Říká se tomu „jižní temperament“.}.
Pokud si myslíte opak, chci vidět metodiku, kterou se to pokusíte změřit.
Tiší Romové bezpochyby existují, akorát o~nich není tolik slyšet,
právě proto, že jsou tiší.

\qaq Proč nechtěj makat jako všichni ostatní?

\qaa Abych mohl odpovědět, podívám se nejprve na ty Romy, kteří pracují.
Pominu zloděje a žebráky, o~těch si myslím to samé jako o~jejich bílých
protějšcích. Pominu ty, kteří vykonávají „normální slušná“ povolání,
jako je třeba řidič autobusu. 

Kdo nám zůstal? Asi tak dvě skupiny. První z~nich jsou různí kopáči,
zedníci a jiné nekvalifikované dělnické profese. Druhá skupina jsou ti Romové,
kteří dělají uklízeče obracejících se rychlíků například ve Vršovicích
nebo na Smíchově. To je velmi nevděčné povolání a~to, že ho dělají, rozhodně
dokazuje, že pracovat chtějí nebo potřebují.

Řekl bych, že ti ostatní se dělí na „nemohou“ a~„nechtějí“.

„Nemohou“, protože nemají odpovídající vzdělání, protože zaměstnavatelé
raději na to samé místo přijmou někoho bílého (kromě toho, že jste Rom,
vás stejně spolehlivě může diskvalifikovat záznam v~rejstříku trestů).

„Nechtějí“, protože vidina toho, že budou dělat celý život uklízečku nebo
kopáče, nezní moc lákavě. Pro nikoho.

Domnívám se též, že mnozí z~těch manuálně pracujících pracují načerno.
Papírově jako nepracují, ve skutečnosti ovšem „makají jak barevní“,
použiju-li ustálené sousloví\footnote{Všimli jste si, kolik ustálených
sousloví je docela rasistických? Už jenom jezdit načerno…}.

\qaq Proč se jich každej zastává a na bílý se sere?

\qaa Tady bych rozhodně neříkal „každej zastává“. Přeformuluji otázku na:
„Proč nikdo neřekne na plnou hubu, co si myslí?“

A tady bych řekl, že se projevuje pokrytectví. Mnoho politiků a novinářů
si určitě myslí nepěkné věci o Romech, ale neodváží se je říct, protože
v jejich odvětvích je silná konkurence a zástupy jiných politiků a jiných
novinářů jen čekají na to, kdy ti první udělají nějakou chybu.
A provolání „Fůůůj! On je rasista!“ se zatím dá dobře použít na likvidaci
konkurence.

Potom zde máme různé socialisty, aktivisty a ochránce lidských práv.
U nich je důvod „obhajoby Romů“ v kontextu jejich víry a světonázoru
pochopitelný. Házení do jednoho pytle se jim (oprávněně) příčí.

To, že je oficiálně slyšet hlavně ta druhá skupina je dáno snahou
o~politickou korektnost\footnote{Proto se taky zrodil termín „nepřizpůsobiví“.}
a již zmíněným pokrytectvím.

A proč se sere na bílý? No, řekl bych, že odpověď je prostá:
Dnes se sere na každého.

\qaq Proč jen sajou prachy ze státu?

\qaa O problému zneužívání sociálních dávek už bylo napsáno dost.
Čtenáře odkážu na článek Ondřeje Lánského
\kOdkaz{Kdo zneužívá sociální dávky}{http://www.denikreferendum.cz/clanek/11822-kdo-zneuziva-socialni-davky} otištěný v deníku Referendum, který tento fenomén dokumentuje řečí přesných čísel. Já sám do té problematiky moc nevidím a zastávám názor, že s tou vší byrokracií okolo by sociální dávky mohli zneužívat leda právníci.

Z vlastní zkušenosti, jako člověk který má čtyři mladší sourozence,
jen řeknu, že mít děti rozhodně není „zdroj příjmů“, ba naopak, je to pěkně
„drahá zábava“. Už jen všechny ty sešity, třídní fondy a podobné věci,
co se platí každý rok v září, jsou celkem pálka. 
Ano, vzdělání je v Čechách (díky bohu) zatím sice zdarma, nicméně
i~tak se prodraží. A~není to jen vzdělání…

A~co jediná účelová dávka specifická pro Romy, ta na podporu středního
vzdělávání Romů\footnote{Negarantuji, že ještě existuje.}?
Zastávám jednoduchý názor — čím více Romů s maturitou tu bude, tím lépe.
Pro všechny.

\qaq Když napadne bílej Roma, tak z~toho nadělaj nevim co, ale když je
to naopak, tak se neděje nic?

\qaa Zde bych neřekl, že se neděje nic. Takové útoky vyvolají obvykle silnou
odezvu u místní komunity. Mnoho z~těch protiromských demonstrací začalo
právě podobným napadením.

Stejně tak velkou odezvu vyvolají rasistické útoky na Romy. To je jedno
z~těch mediálně vděčných témat, které se rozmazává jak rozšlápnutý slimák
dlouho po původní události.

Jinak orgány činné v trestním řízení by se měly snažit v obou případech,
pokud se tak neděje, je něco špatně.

Ale na jedné věci se doufám shodnu i~s rasisty: Házet zápalné lahve do
domu, kde spí děti, je blbost! Blbost, protože kdybych to nazval zvěrstvo,
urážel bych zvěř.

\qaq Nikdo bílej si ke mě nedovolil to, co Cikáni. Vidím partu mladejch
Cikánů a vim, že bude problém. Bojim se jich.

\qaa A tady jsem se dostal k jádru pudla. Těžko můžu chtít například po někom,
koho v dětství třikrát přepadli Cikáni, aby nebyl rasista. Dost nadělají média,
dost nadělají různí Vandasové a spol., ale bohužel je tu stále ta část,
kterou je špatná osobní zkušenost.

Já jsem vyrostl, když to řeknu ošklivě, v „etnicky čisté“ vesnici.
Místní svině byly bíle a místní dobráci taky bílí. Nepovažuju se za rasistu.
Přesto si ovšem například nevybírám tu zkratku, která vede v Holešovicích
skrz místní převážně romskou ubytovnu. {\it Jeden nikdy neví.}

A stejně jako koluje mnoho předsudků o~Romech, kolují i~předsudky o~rasistech.
Takový rasista nemusí být nutně jen manuálně pracující holohlavec v tričku
Thor Steinar. Rasista může být i~normální vzdělaný kravaťák, stejně jako
hezká mladá holka.

Stejně tak bychom si neměli myslet, že to, co říkají rasisté, je vždycky jen lež.
Například tvrzení „ve středověku bylo legální zabít Cikána“ je pravdivé.
Ano, na mnoha místech ve středověku skutečně podobný zákon platil.
Z~toho ovšem rozhodně nevyplývá, že by tomu tak mělo být i~dnes!

Poslední dobou se též začínám přiklánět k~myšlence, že rovnost všech lidí
před zákony je umělá myšlenka. Ale o~to víc bychom si ji měli bránit.
Ne každý stát to totiž má!

Stejně tak jako není dobrý nápad házet do jednoho pytle všechny rasisty,
není dobrý nápad házet do jednoho pytle všechny Romy. Měli bychom je posuzovat
individuálně, člověk od člověka, jako to děláme s bílými.

Tenhle je zloděj, tenhle žebrák, tamten kopáč, tamten řidič autobusu a tamten
hraje v cimbálovce. Ti, co jsem jmenoval, můžou být Romové, ale stejně tak to
mohou být i~bílí.

\startframedtext[width=\textwidth]
{\sc \rm Grammar nazi box}

Slovo Rom (nebo Cikán) se píše s velkým písmenem na začátku, protože se
jedná o příslušníka národa. Stejně tak se píší s~velkým Češi, Poláci či Japonci. 

Oproti tomu se slovo „bílý“ píše s malým písmenem na začátku, protože
se nejedná o příslušníka národa, kmene či obce. 

Pojmenování „grammar nazi“ nemá se skutečnými nacisty nic společného.
Jedná se jen o tradiční (občas posměšné) označení člověka přehnaně lpícího
na pravidlech pravopisu používané v~internetových diskuzích.

\stopframedtext

\kStopPrispevku

\kStrankaIlustrace{n1/romske-otazky.jpg}{To, že je oficiálně slyšet hlavně ta druhá skupina je dáno snahou
o~politickou korektnost. Proto se taky zrodil termín „nepřizpůsobiví“.}

%%% %%% %%%

\prispevek{Jakub Tomaštík}{Vliv masmédií na postoje, názory a hodnoty jedince ve společnosti}

Zadání zní jasně. Tedy do práce.

Co to jsou masmédia? Masová média. Co to jsou média? Z~latinského medius či mediator neboli střední či prostředník. Tedy někdo, kdo stojí mezi, v tomto ohledu, zdrojem a~příjemcem informací. Masová média neboli tedy hromadné sdělovací prostředky.

Jako vše, tak i masmédia, mají svou historii a funkci. Od starověkých Římských řečníků, průběhem času přes církevní hodnostáře, ideologické názory až po dnešní noviny, rozhlas, televizi a hlavně pak internet. Aby tomu nebylo málo, jsou vnímány i jako jakýsi problém. Ptám se, proč? Proč je problémem něco, co je v konceptu skvělá věc s tak lukrativním potenciálem?

Mají vliv na člověka! Nebojím se srovnání s narkotiky. I média mohou totiž krom svého negativního dopadu být návyková. A to nejen u jednotlivců, ale i celých skupin lidí a států. V dnešní době peněz bohužel platí, že kdo ovládá média, ovládá veřejné mínění. Je moc hezké, že lidé věří. Lidé by měli věřit! Měli by však věřit hlavně sami sobě a ne něčemu vnějšímu. Média jsou výtvor lidí, tím pádem za každou jeho formou, kterou jsem již zmiňoval, stojí jeden konkrétní člověk s vlastními cíli, s cíli prodeje a zisku peněz. Jedná se tedy o přímý vliv člověka na člověka skrze média. Média tu slouží pouze jako příjemcovi prodávaný prostředník, který však odebírá zdroji konkrétní identitu. Ponechává zdroj v~anonymitě a~on tak přestává být potencionálním terčem.

Lidé rádi vyhledávají a věří vnějším informacím, protože podle našich společenských norem platí, že kdo není dostatečně a~aktuálně informovaný, je zaostalý. Lidé jsou rádi a snadno ovlivňováni těmito informacemi. Tyto informace, za kterými vždy stojí konkrétní člověk, ovlivňují přímo naše jednání, postoje a názory. Přímo nám říkají, doslova nám vnucují, jak vypadá "ideál krásy", jaké věci jsou ty "nejlepší" na trhu, která značka je "nejkvalitnější", která banka je "nejvstřícnější" či které jídlo je to "nejzdravější". A lidé hltají. Slepě hltají, následují a~skupují jako stádo.

Média mají velký vliv na lidské hodnoty a morálku. Velký a bohužel ve většině případů negativní. Smrt v přímém přenosu, neboli ze zpráv se stává černá kronika. Opravdu je důležité pro normálního člověka vědět, kdo kde ve světě umřel či kde se odehrál jaký masakr? Nejen že to intenzivně zasahuje do soukromí obětí, ale zároveň to v příjemcích vyvolává strach. Strach, se kterým je dále cíleně pracováno. Díky televizi a~internetu se celý svět se všemi katastrofami, válkami a~osudy lidí smrsknul do jedné malé krabičky před námi.

Nejhorší na tom všem však je, že se to děje dnes a denně a člověk tak začíná být na tyto negativity zvyklý, začíná je přijímat jako jakousi samozřejmost, jako věc, která se běžně děje. A to má na nás sakra velký vliv!

Děkuji.

\page

\kStrankaIlustrace{n1/media.jpg}{Nebojím se srovnání s narkotiky. I média mohou totiž krom svého negativního dopadu být návyková.}

%%% %%% %%%
\rubrika{Básně}

\prispevek{Mikoláš Štrajt}{Variace na Vodníka}

\startcolumnset[dvaSloupce]

\startlines
„Ty ses bál?
Já nevěděla, že se bojíš…“ -
„Prý člověk vidí dál,
když takhle nad vodou stojíš.

Já měl té noci divné zdání,
že nejkrásnější z vodaček,
dnes hodlá pojmout za svou paní
malý zelený panáček.“

Kde voda šumí a klokotá,
tam plánuje a skřehotá,
jak pod jezem ji utopí,
až z lodičky ji vyklopí.

A polila mě v mžiku hrůza,
když došlo mi, že přece
tos ty, má drahá můza,
ta nejkrásnější na řece.

Proto já říkal roztřeseně,
že necítím se dnes,
ať radši lodě přeneseme,
jak zrádně vypadá ten jez.

Jak rybka chytlá na udici,
nedá si, nedá milá říci.
Přejet ten jez ji cos nutí,
niť suchá ji není po chuti.

„Vždyť tenhle hravě přejedeme.
Nic nemůže se stát!
Maximálně se vykoupeme.
Nemáš snad vodu rád?“

Však za strašpytla nebudu,
tak lodičku jsme odrazili,
pádla do vody ponořili
a vypluli vstříct osudu.

Tam kde je malá elektrárna vodní,
korytem z betonu musíš plout.
V tom korytě den po dni,
vodáky sráží boční proud.
\stoplines

\column

\startlines
Člověk se prostě rád dívá,
tak na betonovém okraji,
jak už to u jezů bývá,
čumilové tam čekají.
\blank[1*halfline]
Sjeli jsme hladce, nic to není.
Však dole čeká překvapení.
Mohutná vlna, až mrazí,
nás pod hladinu srazí.
\blank[1*halfline]
Nese mě, řeka tiše nese.
Barely naše vodou plují.
Vynořil jsem se v místě, kde se
vody o ostrov rozdělují.
\blank[1*halfline]
Volám, ach volám, svoji paní,
odpověď kloudnou řeka nemá.
Jenom to žabí zakvákání:
„Je má! Je má! Je má!“

Nevesely, truchlivy jsou ty vodní kraje,
kde si starý převozník na námořníka hraje.
V pruhovém trikotu, stařičké pramici,
převáží přes řeku turisty platící.
\blank[1*halfline]
Na břehu koupaliště, tam obvykle k hnutí není.
Ó jaké šťastné náhody!
Právě dnes tam záchranka pořádá cvičení
v tahání lidí z vody.
\blank[1*halfline]
Ředitel záchranky tiše se usmívá,
náhle ho něco znejistí.
„To přece součástí cvičení nebývá,“
říká, když mou milou zajistí.
\blank[1*halfline]
Křísí ji, křísí ji u stánku s Kofolou.
Mrkla! Už zase se hýbá!
Špinavou, odřenou vytáhli mou milou,
ale hlavně je živá!
\blank[1*halfline]
Málem se změnila v mrtvého motýla,
večer pak řekla mi beze všech vytáček,
že potom, co omdlela po ráně do týla,
děsil ji pod vodou zelený panáček!!!
\blank[1*halfline]
{\sc \rm Finis}
\stoplines

\stopcolumnset

\page

\kStrankaIlustrace{n1/vodnik.jpg}{děsil ji pod vodou zelený panáček}

%%% %%% %%%
\rubrika{Komixy}
\prispevek{Anna Benešová}{Život kocoura}
\externalfigure[n1/komix1.png][width=\hsize]
\blank[1*halfline]
\hairline
\blank[1*halfline]
\externalfigure[n1/komix2.png][width=\hsize]

%%% %%% %%%

\hairline
\startlines
% { \tfx
Neslyšný kočkopes \kSep číslo {\os 2013/1} \kSep www: \goto{kockopes.tumblr.com}[url(http://kockopes.tumblr.com)] 
Redakční rada: David Hromádka a Mikoláš Štrajt \kSep Redakční e-mail: neslysny.kockopes@gmail.com
Ilustrace: Dominika Izáková \kSep Korektury: David Hromádka \kSep Sazba: {\tfxx v programu \CONTEXT{}} Mikoláš Štrajt
Neslyšný kočkopes je šířen pod licencí {\em CC-BY-SA (\kOdkaz{Uveďte autora-Zachovejte licenci 3.0 Česko}{http://creativecommons.org/licenses/by-sa/3.0/cz/})}.
Můžete ho {\em šířit} a dále {\em upravovat} pokud {\em uvedete autory} a~{\em zachováte licenci}.
Příjmáme příspěvky do příštího čísla!
% }
\doifmode{economy}{
\hairline
Varování: {\em Čtete úspornou verzi bez ilustrací. Plnou verzi naleznete na \goto{kockopes.tumblr.com}[url(http://kockopes.tumblr.com)] }
}
\stoplines

\stoptext